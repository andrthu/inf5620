\documentclass[11pt,a4paper]{report}

\begin{document}
\begin{center}
\LARGE INF5620 - Compulsory project 1
\\
Andreas Thune
\\
\LARGE
01.10.2015

\end{center}

\textbf{Exercise 2.2}
\\
\\
Want to discretizise the equation (1): $$ \frac{\partial^2 u}{\partial^2 t} + b\frac{\partial u}{\partial t} = \frac{\partial }{\partial x}(q(x,y)\frac{\partial u}{\partial x}) + \frac{\partial }{\partial y}(q(x,y)\frac{\partial u}{\partial y}) + f(x,y,t)$$ on the domain $\Omega=[0,L_x]\times [0,L_y]$, and time in $t\in [0,T]$.
\begin{center}
\Large \textbf{Discretize domain}
\Large


\end{center}

Divide $[0,L_x]$ into $N_x$ intervals $[x_i,x_{i+1}]$, with $i \in \{0,...,N_x-1\}$, with $x_0=0$ and $x_{N_x}=L_x$. Do the same for $[0,L_y]$, so you get $\{y_j \}_{j=0}^{N_y}$, and $[0,T]$ so you get $\{t_n \}_{n=0}^{N_t}$. From now on let $u(x_i,y_j,t_n)=u_{i,j}^n$.

\begin{center}
\Large \textbf{Discretize equation for inner points}
\Large


\end{center}  
I discretize with finite differences term by term. For $ \frac{\partial^2 u}{\partial^2 t}$ I use centred difference: $$\frac{\partial^2 u}{\partial^2 t}(x_i,y_j,t_n) \approx \frac{u_{i,j}^{n+1}-2u_{i,j}^n+u_{i,j}^{n-1}}{\Delta t^2} $$ For $b\frac{\partial u}{\partial t}$ I use forward difference: $$b\frac{\partial u}{\partial t}(x_i,y_j,t_n) \approx b\frac{u_{i,j}^{n+1}-u_{i,j}^n}{\Delta t} $$ For $\frac{\partial }{\partial x}(q(x,y)\frac{\partial u}{\partial x})$ and $ \frac{\partial }{\partial y}(q(x,y)\frac{\partial u}{\partial y})$ I use the normal centred difference for variable coeffichent: $$\frac{\partial }{\partial x}(q(x,y)\frac{\partial u}{\partial x}(x_i,y_j,t_n)) \approx \frac{1}{\Delta x^2}(q_{i+\frac{1}{2},j}(u_{i+1,j}^{n}-u_{i,j}^{n})-q_{i-\frac{1}{2},j}(u_{i,j}^{n}-u_{i-1,j}^{n}))$$ 
$$\frac{\partial }{\partial y}(q(x,y)\frac{\partial u}{\partial y}(x_i,y_j,t_n)) \approx \frac{1}{\Delta y^2}(q_{i,j+\frac{1}{2}}(u_{i,j+1}^{n}-u_{i,j}^{n})-q_{i,j-\frac{1}{2}}(u_{i,j}^{n}-u_{i,j-1}^{n}))$$
Want to find expression for $u_{i,j}^{n+1}$, for calculating inner points. Set $C=\frac{\partial }{\partial x}(q(x,y)\frac{\partial u}{\partial x}) +\frac{\partial }{\partial y}(q(x,y)\frac{\partial u}{\partial y})+f(x,y,t)$. Then : $$ \frac{u_{i,j}^{n+1}-2u_{i,j}^n+u_{i,j}^{n-1}}{\Delta t^2} + b\frac{u_{i,j}^{n+1}-u_{i,j}^n+}{\Delta t} =C $$ $$ \iff u_{i,j}^{n+1}-2u_{i,j}^n+u_{i,j}^{n-1} +\Delta t b(u_{i,j}^{n+1}-u_{i,j}^n)=\Delta t^2 C $$ $$ \iff u_{i,j}^{n+1}(1+\Delta t b)=u_{i,j}^n(2+\Delta t b)-u_{i,j}^{n-1}+\Delta t^2 C$$ $$\iff u_{i,j}^{n+1}=\frac{•u_{i,j}^n(2+\Delta t b)-u_{i,j}^{n-1}+\Delta t^2 C}{(1+\Delta t b)} $$  If we set in expression for $C$ we get (2): $$u_{i,j}^{n+1}=\frac{u_{i,j}^n(2+\Delta t b)-u_{i,j}^{n-1}+\Delta t^2 f_{i,j}^n}{(1+\Delta t b)} $$ $$ + \frac{\Delta t^2}{\Delta x^2}\frac{(q_{i+\frac{1}{2},j}(u_{i+1,j}^{n}-u_{i,j}^{n})-q_{i-\frac{1}{2},j}(u_{i,j}^{n}-u_{i-1,j}^{n}))}{(1+\Delta t b)} $$ $$+ \frac{\Delta t^2}{\Delta y^2}\frac{(q_{i,j+\frac{1}{2}}(u_{i,j+1}^{n}-u_{i,j}^{n})-q_{i,j-\frac{1}{2}}(u_{i,j}^{n}-u_{i,j-1}^{n}))}{(1+\Delta t b)} $$
\begin{center}
\Large \textbf{Discretize first step}
\Large


\end{center}
For the first step we do not know $u_{i,j}^{-1} $, but the initial condition $u(x,y.0)=I(x,y)$ gives us $u_{i,j}^{0}$. The second initial condition is $u_t(x,y,0)=V(x,y)$. To find an expression for $u_{i,j}^{-1}$, we discretize $u_t(x,y,0)=V(x,y)$: $$u_t(x_i,y_j,0) \approx \frac{u_{i,j}^{1}-u_{i,j}^{-1}}{2\Delta t}$$ Set this equal to $V_{i,j}$, and we get an expression for $u_{i,j}^{-1} $: $$\frac{u_{i,j}^{1}-u_{i,j}^{-1}}{2\Delta t} =V_{i,j} \iff u_{i,j}^{-1} = u_{i,j}^{1} -2\Delta t V_{i,j} $$  
Set this into the above result for inner points we get: $$u_{i,j}^{1}(1+\Delta t b)=u_{i,j}^0(2+\Delta t b)-u_{i,j}^{1}+2\Delta t V_{i,j}+\Delta t^2 C $$ $$\iff u_{i,j}^{1}(2+\Delta t b)=u_{i,j}^0(2+\Delta t b)+2\Delta t V_{i,j}+\Delta t^2 C $$ $$\iff u_{i,j}^{1} = \frac{u_{i,j}^0(2+\Delta t b)+2\Delta t V_{i,j}+\Delta t^2 C}{2+\Delta t b} $$
Here $C$ is the decrete version of: $$C=\frac{\partial }{\partial x}(q(x_i,y_j)\frac{\partial u(x_i,y_j,0)}{\partial x}) +\frac{\partial }{\partial y}(q(x_i,y_j)\frac{\partial u(x_i,y_j,0)}{\partial y})+f(x_i,y_j,0)$$
\begin{center}
\Large \textbf{Discretize Neumann conditions}
\Large


\end{center}
Lastly we need to handle the boundary, with homogeneous Neumann conditions. There are three cases:
\\
\\
(i): $(x_i,y_j) \in (0,L_x)\times \{ 0\} \cup (0,L_x)\times \{ L_y\} $.
\\
\\
(ii): $(x_i,y_j) \in \{ 0\}\times(0,L_y)  \cup  \{ L_x\} \times (0,L_y)$.
\\
\\
(iii): $(x_i,y_j) \in \{(0,0),(0,L_y),(L_x,0),(L_x,L_y)  \}$.
\\
\\
For case (i) I only look at points $(x_i,y_0)$. The Neumann condition gives us $u_y(x_i,y_0,t_n)=0$. With the normal discretization of this we get: $$ \frac{u_{i,1}^n-u_{i,-1}^n}{2\Delta y}= 0 \iff u_{i,-1}^n = u_{i,1}^n $$ The only part of the scheme that is affected is $ \frac{\partial }{\partial y}(q(x,y)\frac{\partial u}{\partial y}) $, we get: $$ \frac{1}{\Delta y^2}(q_{i,\frac{1}{2}}(u_{i,1}^{n}-u_{i,0}^{n})-q_{i,-\frac{1}{2}}(u_{i,0}^{n}-u_{i,-1}^{n})) =\frac{1}{\Delta y^2}(q_{i,\frac{1}{2}}(u_{i,1}^{n}-u_{i,0}^{n})+q_{i,-\frac{1}{2}}(u_{i,1}^{n}-u_{i,0}^{n})) $$ $$=\frac{1}{\Delta y^2}(u_{i,1}^{n}-u_{i,0}^{n})(q_{i,\frac{1}{2}}+q_{i,-\frac{1}{2}}) $$ Since $q_{i,-\frac{1}{2}} $ is evaluating q outside of our domain, we make the following ansatz: $q_{i,-\frac{1}{2}} \approx q_{i,\frac{1}{2}} \approx q_{i,0}$. This gives: $$\frac{\partial }{\partial y}(q(x_i,0)\frac{\partial u(x_i,0,t_n)}{\partial y}) \approx \frac{2(u_{i,1}^{n}-u_{i,0}^{n})q_{i,0}}{\Delta y^2} $$ For $y=L_y$, we get the following by same arguments: $$\frac{\partial }{\partial y}(q(x_i,L_y)\frac{\partial u(x_i,y_{N_y},t_n)}{\partial y}) \approx \frac{2(u_{i,N_y-1}^{n}-u_{i,N_y}^{n})q_{i,N_y}}{\Delta y^2} $$ For case (ii), we get the same, but now we have: $$u_x(x_0,y_j,t_n)=u_x(x_{N_x},y_j,t_n)=0$$ By the same approach as for case (i), we get $$\frac{\partial }{\partial x}(q(0,y_j)\frac{\partial u(0,y_j,t_n)}{\partial x}) \approx \frac{2(u_{1,j}^{n}-u_{0,j}^{n})q_{0,j}}{\Delta x^2} $$ and $$\frac{\partial }{\partial x}(q(L_x,y_j)\frac{\partial u(x_{N_x},y_j,t_n)}{\partial x}) \approx \frac{2(u_{N_x-1,j}^{n}-u_{N_x,j}^{n})q_{N_x,j}}{\Delta x^2} $$ For case (iii), we get both case (i) and (ii). This means we already have the expressions needed to  explicitly find u for points in (iii). If you want the formulas for the boundary points, you need to substitute the expressions for the x and y derivatives in (2) with the ones we have found above.
\\
\\
\\
\\
\\
\textbf{Exercise 3.1}
\\
\\
Want to prove that a constant solution also solves the discrete equation: $$ \frac{u_{i,j}^{n+1}-2u_{i,j}^n+u_{i,j}^{n-1}}{\Delta t^2} + b\frac{u_{i,j}^{n+1}-u_{i,j}^n}{\Delta t} $$ $$ = \frac{1}{\Delta x^2}(q_{i+\frac{1}{2},j}(u_{i+1,j}^{n}-u_{i,j}^{n})-q_{i-\frac{1}{2},j}(u_{i,j}^{n}-u_{i-1,j}^{n}))$$ $$ +\frac{1}{\Delta y^2}(q_{i,j+\frac{1}{2}}(u_{i,j+1}^{n}-u_{i,j}^{n})-q_{i,j-\frac{1}{2}}(u_{i,j}^{n}-u_{i,j-1}^{n}))$$ A constant solution $u(x,y,t)=c$ means that $u_{i,j}^n=c$. This means that $$ \frac{u_{i,j}^{n+1}-2u_{i,j}^n+u_{i,j}^{n-1}}{\Delta t^2} + b\frac{u_{i,j}^{n+1}-u_{i,j}^n}{\Delta t} = \frac{c-2c+c}{\Delta t^2} + b\frac{c-c}{\Delta t} = 0$$ and: $$ \frac{1}{\Delta x^2}(q_{i+\frac{1}{2},j}(u_{i+1,j}^{n}-u_{i,j}^{n})-q_{i-\frac{1}{2},j}(u_{i,j}^{n}-u_{i-1,j}^{n}))$$ $$ = \frac{1}{\Delta x^2}(q_{i+\frac{1}{2},j}(c-c)-q_{i-\frac{1}{2},j}(c-c))=0$$ and: $$ \frac{1}{\Delta y^2}(q_{i,j+\frac{1}{2}}(u_{i,j+1}^{n}-u_{i,j}^{n})-q_{i,j-\frac{1}{2}}(u_{i,j}^{n}-u_{i,j-1}^{n}))=0$$ This means that the discrete equation is satisfied when u is constant.
\\
\\
\textbf{Exercise 3.1.4: Making bugs}
\\
To make it simple, all my bugs are made in the first step, and are shown in the code. 
\\
\\
(1): The first bug is to not implement the initial condition correctly. This causes errors.
\\
\\
(2): The second bug, is to change a sign in the scheme. This does not cause errors, since the terms I subtract are both zero.
\\
\\
(3): The third bug is also a changed sign, but this time the two non-zero terms are added instead of subtracted, and we therefore get errors.
\\
\\
(4): The fourth bug is changing the q function in the in the scheme. This does not create errors, since the constant solution does not depend on q because q is multiplied by zero.
\\
\\
(5): Use current timestep to calculate one of the boarders instead of the previous one. This also causes no error.
\\
\\
\textbf{Exercise 3.3: Plug wave solution}
\\
When I tested that i should get exact solution, I made sure that the wave did not hit the boundary at the end time. This means that the exact solution at time t is given by Dalambert formula $u(x,y,t)=\frac{1}{2}(I(x-t,y)+I(x+t,y))$. To check that I got exact solution look in code.
\\
\\
\textbf{Exercise 3.4: Standing undamped waves}
\\
Everything is in the code.
\\
\\
\textbf{Exercise 3.6: Manufactured solution}
\\
Everything in code. Used sympy to find sorce-term. Did not get exactly 2 in convergence rate, but closer to 2 than 1.
\end{document}
